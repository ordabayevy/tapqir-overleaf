\section{Introduction}

Single-molecule fluorescence microscopy multi-wavelength colocalization methods (CoSMoS) are an important method for elucidating the mechanisms of complex biochemical processes \textit{in vitro}. CoSMoS is extensively used by labs that specialize in the technique and is increasingly being adopted by non-specialist labs as well. With the availability of good commercial microscopes, data analysis methodology is the major challenge in adoption and use of the CoSMoS technique.

Analysis of CoSMoS data is an intrinsically challenging problem. Although CoSMoS images are conceptually simple -- they consist only of diffraction-limited fluorescent spots collected in several wavelength channels -- efficient analysis of the images, reliable extraction of kinetic data, and fitting to model biochemical mechanisms are inherently challenging for several reasons. First, the number of photons emitted by a single fluorophore is limited by fluorophore photobleaching. Consequently, in many CoSMoS experiments one must work at the lowest feasible excitation power in order to maximize the duration of experimental recordings and to capture relevant kinetics. Achieving higher time resolution divides the number of emitted photons between a larger number of images. The required high concentrations of fluorescently tagged molecules create significant background noise \citep{peng_breaking_nodate, van_oijen_single-molecule_2011}, even with zero-mode waveguide instruments \citep{chen_high-throughput_2014}. These technical difficulties result in CoSMoS images that frequently have low signal-to-noise (S/N) ratios, making discrimination of real fluorescent spots from noise a major challenge. Second, there are transient non-specific interactions of the binder molecule with the surface of the microscope slide. These can result both in false positive colocalization detection when the binder molecule randomly lands near the target molecule and false negative detection when non-specific binding overlaps with the binder molecule at the target location by deforming its shape. These leads to erroneous estimates of the kinetic parameters and can be accounted for by joint analysis of the negative control locations and explicit modeling of non-specifically bound spots. Third, the kinetic analysis of thermally driven single molecule reactions is inherently statistical. Even if image analysis is perfectly reliable, statistical bias in the underlying experimental data (e.g., under-counting short binding events) can lead to erroneous mechanistic conclusions. In real experiments, where image analysis is necessarily less than perfect, there is a need to quantitatively account for the uncertainties in CoSMoS image data to deduce molecular mechanisms.

Current spot detection methods are based on extracting quantitative features of the spot (e.g., intensity profile, size, shape, and distance to target) from the 2-D image and then identifying spots using manually chosen thresholds for these parameters \citep{friedman_cosmos_analysis_2015, smith_automated_2019}. These image analysis based methods are more accurate than previous approaches in part because they make use of information contained in the 2-D images that are disregarded when only the integrated intensity is used \citep{friedman_cosmos_analysis_2015}. While these methods greatly improve CoSMoS data analysis, they nevertheless suffer from significant deficiencies. First, current methods require subjective choice of user-set thresholds for spot amplitude, diameter and proximity. These settings significantly affect error rates and no objective method to select them currently exists, making the approach non-robust and overly complicated. Second, it is unsatisfying that existing analysis methods are discontinuous and there is a loss of information at each step of the analysis. Image classification is performed on extracted features of the spot and not the raw 2-D images themselves. Furthermore, image classification step produces only a binary output (spot present or absent); they do not output the probability of spot presence in marginal images, a critical feature in the analysis of low S/N data. Kinetic analysis, in turn, is based on the binary output of the image classification step further reducing the amount of information transmitted from the raw images. Third, it is nontrivial to incorporate prior knowledge (e.g., colocalization accuracy, size of the spot, frequency of non-specific binding) into these analysis methods.

Here we present novel, objective, statistically based approach for analysis and interpretation of CoSMoS data. Our new method: 1) maximizes extraction of useful results from data (particularly at the low S/N inherent to many CoSMoS experiments) by analyzing raw images, not just fluorescence intensities; 2) applies single global probabilistic model which allows to directly infer model parameters from CoSMoS image data and priors; 3) explicitly models non-specific interaction of the binder molecule with the surface which allows to jointly analyze negative control dataset; 4) uses Gamma distribution as a more realistic intensity noise model rather than Gaussian distribution; 4) has a flexible framework that can naturally be extended to kinetic models and multi-wavelength analysis. Our new analysis method will increase the accessibility of the CoSMoS method to investigators that have important biological problems to solve but are not able to develop their own data analysis methods or to reliably use the current data analysis approaches which require extensive parameter tweaking. The method developed here eliminates dependence on subjective parameter adjustments and include built-in tools for statistical assessment of the validity of the results as essential steps in the analysis process.

\begin{comment}
Coordinate optimization of CoSMoS image and mechanistic/kinetic models. We propose analyzing CoSMoS data using a novel approach in which we co-optimize inference of both CoSMoS image classifications and the underlying molecular kinetic mechanism simultaneously, by integrating the Bayesian image and kinetic analysis together under a single global probabilistic model. This approach is inspired by previously successful applications of Hidden Markov Models (HMMs) to elucidate kinetic mechanisms and derive realistic rate constants from scalar time series data (e.g., from smFRET and single-channel electrophysiology experiments such as in refs. 11–16). To our knowledge this approach has not previously been extended to inferring biochemical reaction mechanisms directly from 2-D single-molecule images.
 
The proposed research will facilitate the application of CoSMoS to more complex biological systems and experiments. As the CoSMoS technique has become better established, its applications have become increasingly complex. In some published and unpublished studies we now routinely work at labeled molecule concentrations approaching 1 µM, perform three- and four-color experiments, simultaneously analyze binding to multiple types of separately identified target molecules in the same microscope field of view, and examine in vitro systems of biochemical complexity up to and including whole cell extracts. Such complex experiments present increasingly difficult problems of data analysis. The proposed research (Aim 2) will greatly aid in the interpretation of the complex data sets produced by the current generation of CoSMoS experiments by integrating the capability to do the time series analyses and kinetic modeling needed for mechanistic interpretation of the data. They will allow accurate elucidation of complex molecular mechanisms and efficient accommodation of experimental imperfections like photobleaching and incomplete labeling.

Background: Differences between analysis of CoSMoS data and analysis of super-resolution or single-molecule tracking data. Super-resolution techniques like PALM and STORM45–47 are superficially similar to CoSMoS, as they also involve localizing spots of single-molecule fluorescence. The same is true for live-cell single-molecule fluorescence detection/tracking methods (e.g., refs. 48–51). However, these applications differ from CoSMoS in fundamental ways: 1) PALM and STORM use comparatively high S/N, short duration imaging to maximize the spatial precision of spot localization. CoSMoS is the opposite; much poorer spatial precision is needed but long duration recordings of individual molecules (with consequent lower S/N) are required to provide information about kinetic processes. 2) CoSMoS image analysis is predicated on making use of prior information about the surface locations of target molecules; no such prior information exists in super-resolution or tracking applications. 3) Unlike CoSMoS, neither super-resolution nor tracking experiments are directly tied to underlying biochemical kinetic reaction schemes. Thus, while there has been extensive development of analytical methods for super-resolution and tracking (including some use of Bayesian approaches7–10,52,53), optimal analysis of CoSMoS data will require specifically tailored methodology.

Preliminary results 3: 2-D image-based CoSMoS data analysis. (Ref. 6) The most basic task in CoSMoS data analysis is spot discrimination. The goal is to acquire information at each time point about whether a binder molecule fluorescence spot is observed at the image position of a target molecule (e.g., whether a co-localized green RNA polymerase is observed at the surface location of a blue DNA spot in Figure 1). The discrimination methods that are commonly used (including in earlier work from the Gelles lab26,29,36,38,57,59,60) involve some combination of purely subjective inspection of the microscope images and an objective method based on integrating the binder fluorescence intensity over small regions of the image (typically squares ~0.4 μm on a side) centered on the location of the target molecule and then using crossings of set intensity threshold(s) to score binder molecule arrival and departure. For reasons described in detail elsewhere6 , these approaches are both imprecise and inaccurate, particularly when applied to the low S/N images often used. A fundamental problem with these methods is that even their objective and quantitative components rely on the integrated intensity and thus discard spatial information contained in the images that can and should be used to objectively inform decisions about spot presence.
\end{comment}
\section{Introduction}

Single-molecule fluorescence microscopy multi-wavelength co-localization methods (CoSMoS) are an important method for elucidating the mechanisms of complex biochemical processes \textit{in vitro}. CoSMoS is extensively used by labs that specialize in the technique and is increasingly being adopted by non-specialist labs as well. With the availability of good commercial microscopes, data analysis methodology is the major challenge in adoption and use of the CoSMoS technique.

Analysis of CoSMoS data is an intrinsically challenging problem. Although CoSMoS images are conceptually simple -- they consist only of diffraction-limited fluorescent spots collected in several wavelength channels -- efficient analysis of the images, reliable extraction of kinetic data, and fitting to model biochemical mechanisms are inherently challenging for several reasons. First, the number of photons emitted by a single fluorophore is limited by fluorophore photobleaching. Consequently, in many CoSMoS experiments one must work at the lowest feasible excitation power in order to maximize the duration of experimental recordings and to capture relevant kinetics. Achieving higher time resolution divides the number of emitted photons between a larger number of images. The required high concentrations of fluorescently tagged molecules create significant background noise \citep{Peng2018-ge,Van_Oijen2011-ig}, even with zero-mode waveguide instruments \citep{chen_high-throughput_2014}. These technical difficulties result in CoSMoS images that frequently have low signal-to-noise (S/N) ratios, making discrimination of real fluorescent spots from noise a major challenge. Second, there are transient non-specific interactions of the binder molecule with the surface of the microscope slide. These can result both in false positive co-localization detection when the binder molecule randomly lands near the target molecule and false negative detection when non-specific binding overlaps with the binder molecule at the target location by deforming its shape. These leads to erroneous estimates of the kinetic parameters and can be accounted for by joint analysis of the negative control locations and explicit modeling of non-specifically bound spots.

All of the current analysis methods involve a two-step process. In the first co-localized spot detection step, binary information at each time point about whether a binder molecule fluorescence spot is observed at the image position of a target molecule is obtained. In the second step, kinetic parameters are extracted from on- and off- dwell times. Earlier spot detection methods are based on integrating the binder fluorescence intensity over small regions of the image (typically squares $\sim$0.4 $\mu$m on a side) centered on the location of the target molecule and then using crossings of set intensity threshold(s) to score binder molecule arrival and departure. For reasons described in detail elsewhere \citep{Friedman2015-nx}, these approaches are both imprecise and inaccurate, particularly when applied to the low S/N images often used. A fundamental problem with these methods is that they rely on the integrated intensity and thus discard spatial information contained in the images that can and should be used to objectively inform decisions about spot presence.

Current spot detection methods are based on extracting quantitative features of the spot (e.g., intensity profile, size, shape, and distance to target) from the 2-D image and then identifying spots using manually chosen thresholds for these parameters \citep{Friedman2015-nx, Smith2019-ns}. These image analysis based methods are more accurate than previous approaches in part because they make use of information contained in the 2-D images that are disregarded when only the integrated intensity is used \citep{friedman_cosmos_analysis_2015}. While these methods greatly improve CoSMoS data analysis, they nevertheless suffer from significant deficiencies. First, current methods require subjective choice of user-set thresholds for spot amplitude, diameter and proximity. These settings significantly affect error rates and no objective method to select them currently exists, making the approach non-robust and overly complicated. Second, it is unsatisfying that existing analysis methods are discontinuous and there is a loss of information at each step of the analysis. Image classification is performed on extracted features of the spot and not the raw 2-D images themselves. Furthermore, image classification step produces only a binary output (spot present or absent); they do not output the probability of spot presence in marginal images, a critical feature in the analysis of low S/N data. Kinetic analysis, in turn, is based on the binary output of the image classification step further reducing the amount of information transmitted from the raw images. Third, it is nontrivial to incorporate prior knowledge (e.g., colocalization accuracy, size of the spot, frequency of non-specific binding) into these analysis methods.

To solve the problems mentioned above here we adopt Bayesian inference approach which provides natural and principled way of combining prior information with data, within a solid statistical framework. Bayesian inference relies on Bayes' theorem to deduce posterior distribution of latent variables conditional on data. We define a probabilistic generative model which can be interpreted as a causal process by which the observed data is generated. In our case observed data corresponds to an image at a target site (matrix of pixel intensities). The observed images are generated according to the likelihood function (noise model) for different settings of latent variables such as background intensity, number of spots in the image, position, shape, and intensity of each individual spot. The latent variables, in turn, are sampled from prior distribution. Prior distributions can be used to embed into our expertise knowledge into the model such as the distribution of the position of on-target fluorescent spots or they can represent global parameters of the model as in the case with average probability of observing a spot in the image. We can represent this problem using a graphical model of the form shown in Figure. It provies inferences that are conditional on data and are exact. It provides interpretable answers, such as. It provides a convenient setting for a wide range of models, such as hierarchical and missing data problems

In this article, we present novel, objective, statistically based approach for analysis and interpretation of CoSMoS data. Our new method: 1) provides uncertainties for all of the model parameters including a spot probability estimate for each image (not merely a Boolean spot/no spot classification) 2) maximizes extraction of useful information from data (particularly at the low S/N inherent to many CoSMoS experiments) by analyzing raw images, not just fluorescence intensities; 2) defines a single global probabilistic model which allows to directly infer model parameters from CoSMoS image data; 3) explicitly models non-specific interaction of the binder molecule with the surface which allows to jointly analyze negative control dataset; 4) uses Gamma distribution as a more realistic intensity noise model rather than Gaussian distribution; 5) has a flexible framework that can naturally be extended to kinetic models and multi-wavelength analysis. Our new analysis method will increase the accessibility of the CoSMoS method to investigators that have important biological problems to solve but are not able to develop their own data analysis methods or to reliably use the current data analysis approaches which require extensive parameter tweaking. The method developed here eliminates dependence on subjective parameter adjustments and include built-in tools for statistical assessment of the validity of the results as essential steps in the analysis process.
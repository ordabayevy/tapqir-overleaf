\section{Introduction}

A central concern of modern biology is understanding at the molecular level the chemical and physical mechanisms by which protein and nucleic acid macromolecules  perform essential cellular functions.  The operation of many such macromolecules requires that they work not as isolated molecules in solution but as components of dynamic molecular complexes that self-assemble and change structure and composition as they function.  For more than  two decades, scientists have successfully explored the molecular mechanisms of many such complex and dynamic systems using multi-wavelength single molecule fluorescence methods such as smFRET (single-molecule fluorescence resonance energy transfer) \cite{Roy2008-fo} and single-molecule co-localization methods like CoSMoS (co-localization single molecule spectroscopy) \cite{Larson2014-os, Van_Oijen2011-ig}.

In its simplest implementation, CoSMoS is a technique to visualize transient interactions between individual molecules.  Dye-labeled protein or nucleic acid target molecules are immobilized on a slide surface and TIRF (total internal reflection fluorescence) microscopy is employed to detect co-localizations between those targets and other dye-labeled molecules that might bind those targets during a biochemical reaction.  The CoSMoS method has been used for elucidating the mechanisms of complex biochemical processes \textit{in vitro}. Examples include cell cycle regulation \cite{Lu2015-eu}, ubiquitination and proteasome-mediated protein degradation \cite{Lu2015-jq}, DNA replication \cite{Geertsema2014-bt,Ticau2015-ib}, transcription \cite{Zhang2012-no,Friedman2012-if,Friedman2013-sf}, micro-RNA regulation \cite{Salomon2015-kq}, pre-mRNA splicing \cite{Shcherbakova2013-bi, Krishnan2013-fy, Warnasooriya2014-ls}, ribosome assembly \cite{Kim2014-zc}, translation \cite{Wang2015-tt,Tsai2014-mi,OLeary2013-wo}, signal recognition particle-nascent protein interaction \cite{Noriega2014-vj}, and cytoskeletal regulation \cite{Smith2013-qj,Breitsprecher2012-mj}. CoSMoS is extensively used by labs that specialize in the technique and is increasingly being adopted by non-specialist labs as well. With the availability of good commercial TIRF microscopes, data analysis methodology is a major challenge in adoption and use of the CoSMoS technique.

Analysis of CoSMoS data is an intrinsically challenging and time-consuming problem. Although CoSMoS images are conceptually simple -- they consist only of diffraction-limited fluorescent spots collected in several wavelength channels -- efficient analysis of the images, reliable extraction of kinetic data, and fitting to model biochemical mechanisms are inherently challenging for several reasons. First, the number of photons emitted by a single fluorophore is limited by fluorophore photobleaching. Consequently, in many CoSMoS experiments one must work at the lowest feasible excitation power in order to maximize the duration of experimental recordings and to capture relevant kinetics. Achieving higher time resolution divides the number of emitted photons between a larger number of images. The required concentrations of fluorescently tagged molecules can sometimes create significant background noise \cite{Peng2018-ge, Van_Oijen2011-ig}, even with zero-mode waveguide instruments \cite{Chen2014-jd}. These technical difficulties result in CoSMoS images that frequently have low signal-to-noise (S/N) ratios, making discrimination of real fluorescent spots from noise a major challenge. Second, there are transient non-specific interactions of the binder molecule with the surface of the microscope slide, and these non-specific interactions can give rise to both false positive and false negative detections. Additional measurements and modeling of these non-specific surface interactions must be employed to avoid errors in derived kinetic parameters or inferred mechanisms.

All of the current CoSMoS analysis methods involve a two-step process. In the first co-localized spot detection step, binary information at each time point about whether a binder molecule fluorescence spot is observed at the image position of a target molecule is obtained. In the second step, kinetic parameters are extracted from the on- and off- dwell time distributions. Some spot detection methods are based on integrating the binder fluorescence intensity over small regions of the image (typically squares \SI{0.4}[\sim]{\um} on a side) centered on the location of the target molecule, and then using crossings of set intensity threshold(s) to score binder molecule arrival and departure. For reasons described in detail elsewhere \cite{Friedman2015-nx}, these approaches are both imprecise and inaccurate, particularly when applied to the low S/N images often used. A fundamental problem with these methods is that they rely on the integrated intensity and thus discard spatial information contained in the images that can and should be used to objectively inform decisions about spot presence.

This work will increase the accessibility of the CoSMoS method to investigators that have important biological problems to solve, but are not able to develop their own data analysis methods or to reliably use the current data analysis approaches which require parameter tweaking. The method developed here eliminates this dependence on subjective parameter adjustments and includes built-in tools for statistical assessment of the validity of the results as essential steps in the analysis process.


Current spot detection methods produce binary outputs by using a bandpass-filter set by a user-specified intensity threshold \cite{Friedman2015-nx, Smith2019-yb}. Unlike previous methods, this approach discriminates authentic fluorescence spots from fluctuations in background fluorescence in a probabilistic manner and assigns probabilities of belonging to each class.
\section{Introduction}

Single-molecule fluorescence microscopy multi-wavelength colocalization methods (called “CoSMoS” in this application for short) are an important method for elucidating the mechanisms of complex biochemical processes in vitro. CoSMoS is extensively used by labs that specialize in the technique and is increasingly being adopted by non-specialist labs as well. With the availability of good commercial microscopes, data analysis methodology is the major challenge in adoption and use of the CoSMoS technique. Methods used by others analyze only integrated fluorescence spot intensities and thus disregard additional spatial information contained in the 2-D image data. Current analytical methods involve subjective steps that are labor-intensive and interfere with valid statistical assessment. The proposed research will develop and implement novel, objective, statistically based approaches for analysis and interpretation of CoSMoS data. These new methods will: 1) maximize extraction of useful results from data (particularly at the low S/N inherent to many CoSMoS experiments) by analyzing images, not just fluorescence intensities; 2) integrate image analysis with kinetic analysis; 3) improve the reliability of conclusions based on CoSMoS measurements and provide tools to quantitatively assess reliability; 4) implement a systematic approach for developing and testing the validity of the kinetic models that are needed for deriving conclusions about molecular mechanisms from CoSMoS data.

Analysis of CoSMoS data is an intrinsically challenging problem. To minimize photobleaching, CoSMoS experiments often require low excitation intensities that produce data with low S/N. In addition, reactions of single molecules are thermally-driven stochastic processes, the interpretation of which is inherently statistical. Finally, CoSMoS is premised on knowing the locations of surface-tethered target molecules, and this prior information needs to be incorporated into the analysis in a statistically valid way. The proposed research will increase the accessibility of the CoSMoS method to investigators that have important biological problems to solve but are not able to develop their own data analysis methods or to reliably use the current data analysis approaches which require extensive parameter tweaking. The methods developed here will minimize or eliminate dependence on subjective parameter adjustments and include built-in tools for statistical assessment of the validity of the results as essential steps in the analysis process. We will incorporate the methods we develop in a documented, open-source analysis software package that will be usable by less-expert scientists and provide a path for accommodating future improvements and models developed by us and others.

The proposed research will provide more efficient ways to analyze large CoSMoS data sets. Technological developments such as faster, larger cameras \citep{huang_particle_nodate}, availability of fluorescent dyes with dramatically improved photostability (which increase the amount of data from a single experiment), and microscopes that efficiently collect single-molecule data at more wavelengths simultanously \citep{friedman_viewing_2006} have all increased the sizes of CoSMoS datasets. Gelles lab studies of transient molecular binding events at high frame rates have produced >1 TB image data per experiment. Increasing data set sizes make it infeasible to use existing analytical methods. The proposed methods will eliminate the need for subjective image inspection and minimize the manual work required for CoSMoS data analysis. The algorithms are highly parallelizable and will be implemented so that they can be efficiently run on compute clusters and graphics processing units (GPUs).

The proposed research will facilitate the application of CoSMoS to more complex biological systems and experiments. As the CoSMoS technique has become better established, its applications have become increasingly complex. In some published and unpublished studies we now routinely work at labeled molecule concentrations approaching 1 µM, perform three- and four-color experiments, simultaneously analyze binding to multiple types of separately identified target molecules in the same microscope field of view, and examine in vitro systems of biochemical complexity up to and including whole cell extracts. Such complex experiments present increasingly difficult problems of data analysis. The proposed research (Aim 2) will greatly aid in the interpretation of the complex data sets produced by the current generation of CoSMoS experiments by integrating the capability to do the time series analyses and kinetic modeling needed for mechanistic interpretation of the data. They will allow accurate elucidation of complex molecular mechanisms and efficient accommodation of experimental imperfections like photobleaching and incomplete labeling.
\section*{Methods}

\subsection*{Probabilistic model for the CoSMoS image data} 

Our intent is to model CoSMoS image data by accounting for the most significant physical aspects of the image generation process, in particular a realistic description of photon noise, using model parameters that likewise reflect underlying physical factors known to affect the experimental setup. We use a mixture model to detect binder molecules that are present in the image and to distinguish between specifically and non-specifically bound molecules. An extended version of the graphical representation of the model for CoSMoS data that includes probability distributions is shown Figure S1A. The corresponding generative model represented as pseudocode is shown in Algorithm 1. Below we describe the model in detail starting with the observed data and the likelihood function and then proceed with model parameters and their prior distributions.

\subsubsection*{Observed data and likelihood function}

Microscope image preprocessing gives a list of drift-corrected locations of the target molecules in each frame with sub-pixel accuracy. Based on this information, we select $P \times P$-pixel AOIs centered with pixel resolution at the target location. The observed data ($D$) thus consists of a set of $P \times P$ grayscale values of pixel intensity as measured by the camera in arbitrary units, collected at $N$ number of AOI sites for a range of $F$ number of frames:
%
\begin{gather*}
    D \in \mathbb{R}_{>0}^{\mathsf{AOI}[N] \times \mathsf{frame}[F] \times \mathsf{pixelX}[P] \times \mathsf{pixelY}[P]} \\
    x_c \in [-1/2, 1/2]^{\mathsf{AOI}[N] \times \mathsf{frame}[F]} \\
    y_c \in [-1/2, 1/2]^{\mathsf{AOI}[N] \times \mathsf{frame}[F]} \\
\end{gather*}

\noindent
where $x_c$ and $y_c$ are the coordinates of the target molecule relative to the center of the AOI. While experimental intensity measurements are integers we treat them as continuous values in our analysis.

We model the data $D$ as the sum of a fixed photon-independent offset $\delta$ introduced by the camera and the noisy photon-dependent pixel intensity values $I$.
%
\begin{gather*}
    \delta \in \mathbb{R}_{>0}^{\mathsf{AOI}[N] \times \mathsf{frame}[F] \times \mathsf{pixelX}[P] \times \mathsf{pixelY}[P]} \\
    I \in \mathbb{R}_{>0}^{\mathsf{AOI}[N] \times \mathsf{frame}[F] \times \mathsf{pixelX}[P] \times \mathsf{pixelY}[P]} \\ 
    D = \delta + I
\end{gather*}

In our model, each pixel in the photon-dependent image $I$ has a  variance which is equal to  the mean intensity $\mu^I$ of that pixel multiplied by the gain $g$. This formulation accounts for both photon shot noise and additional noise introduced by EMCCD camera amplification (ref). We model the intensity $I$ using a continuous Gamma distribution as the likelihood function:
%
\begin{gather*}
    I \sim \mathbf{Gamma} (\mu^I, \sqrt{\mu^I \cdot g})
\end{gather*}

The parameterization of the Gamma distribution (and all other distributions we use in this work) is given in Table S2.

\subsubsection*{Image model}

The mean of the photon-dependent pixel intensity $\mu^I$ is represented  as the sum of a background intensity $b$ and the intensities from fluorescence spots modeled as  2-dimensional Gaussians $\mu^S$ 
%
\begin{gather*}
    \mu^I \in \mathbb{R}_{>0}^{\mathsf{AOI}[N] \times \mathsf{frame}[F] \times \mathsf{pixelX}[P] \times \mathsf{pixelY}[P]} \\
    b \in \mathbb{R}_{>0}^{\mathsf{AOI}[N] \times \mathsf{frame}[F]} \\
    m \in \{ 0, 1 \}^{\mathrm{spot}[K] \times \mathsf{AOI}[N] \times \mathsf{frame}[F] } \\
    \mu^I = b + \sum{\mathsf{spot}[K]} m \odot{} \mu^S \\
\end{gather*}

\noindent
where $\odot$ is an element-wise product operation. For simplicity we allow at most $K$ spots in each frame of each AOI. The presence of a given spot in the image is encoded in the binary spot existence parameter $m$, where $m = 1$ when the corresponding spot is present and $m = 0$ when it is absent. The background $b$ is local to each frame in each AOI and is restricted to positive values.



The intensities for a 2-dimensional Gaussian spot centered at ($x$, $y$) are defined for each pixel coordinate ($i$, $j$):
%
\begin{gather*}
    i \in \{-(P-1)/2, \dots, (P-1)/2\}^{\mathsf{pixelX}[P]} \\
    j \in \{-(P-1)/2, \dots, (P-1)/2\}^{\mathsf{pixelY}[P]} \\
    \mu^S \in \mathbb{R}_{>0}^{\mathsf{spot}[K] \times \mathsf{AOI}[N] \times \mathsf{frame}[F] \times \mathsf{pixelX}[P] \times \mathsf{pixelY}[P]}  \\
    \mu^S = \dfrac{h}{2 \pi \cdot w^2} \exp{\left( -\dfrac{(i-x)^2 + (j-y)^2}{2 \cdot w^2} \right)}
\end{gather*}

\noindent
with parameters total integrated intensity $h$, width $w$, and center ($x$, $y$). Coordinates $x$, $y$, $i$, and $j$ are taken relative to the AOI center. 
%
\begin{gather*}
    h \in \mathbb{R}_{>0}^{\mathsf{spot}[K] \times \mathsf{AOI}[N] \times \mathsf{frame}[F]} \\
    w \in \mathbb{R}_{>0}^{\mathsf{spot}[K] \times \mathsf{AOI}[N] \times \mathsf{frame}[F]} \\
    x \in \mathbb{R}^{\mathsf{spot}[K] \times \mathsf{AOI}[N] \times \mathsf{frame}[F]} \\
    y \in \mathbb{R}^{\mathsf{spot}[K] \times \mathsf{AOI}[N] \times \mathsf{frame}[F]} \\
\end{gather*}



To discriminate between spots from target-specific and target-nonspecific binding, we use the index parameter $\theta$ to specify the index of the target-specific spot when it is present and use zero when no target-specific spot is present.
%
\begin{gather*}
    \theta \in \{ 0, 1, \dots, K \}^{ \mathsf{AOI}[N] \times \mathsf{frame}[F] } \\
\end{gather*}

Probability of target-specific spot presence is calculated as

\begin{gather*}
    p(\mathsf{specific}) = p(\theta > 0)
\end{gather*}

\subsubsection*{Prior distributions}

The prior distributions for the unobserved model parameters are detailed below and illustrated in Figure S1. Unless otherwise indicated we assume largely uninformative priors (such as the Half-Normal distribution with large mean). Parameters of prior distributions can be changed if necessary.
%
\begin{enumerate}
    \item Background intensity is modeled using a hierarchical Gamma prior
%
\begin{gather*}
    \mu^b \in \mathbb{R}_{>0}^{\mathsf{AOI}[N]} \\
    \sigma^b \in \mathbb{R}_{>0}^{\mathsf{AOI}[N]} \\
    b \sim \mathbf{Gamma}(\mu^b, \sigma^b)
\end{gather*}

\noindent
where the mean $\mu^b$ and standard deviation $\sigma^b$ of the background intensity describe the irregularity in the background intensity in time and across the field of view of the microscope. Hyperpriors for $\mu^b$ and $\sigma^b$ are uninformative
%
\begin{gather*}
    \mu^b \sim \mathbf{HalfNormal}(1000) \\
    \sigma^b \sim \mathbf{HalfNormal}(100)
\end{gather*}

\item The prior distribution for the index of the target-specific spot $\theta$ is modeled hierarchically in terms of the average specific binding probability $\pi$. The probability that $\theta = 0$ is equal to the probability of no specifically bound spot being present (i.e., $1-\pi$). Since spot indices are arbitrarily assigned, the probability that the specifically bound molecule is present is equally split between those indices (i.e., $\frac{\pi}{K}$). We represent the prior for $\theta$ as a Categorical distribution of the following form
%
\begin{gather*}
    \theta \sim \mathbf{Categorical}\left(1 - \pi, \frac{\pi}{K}, \dots, \frac{\pi}{K}\right)
\end{gather*}

\noindent
The average target-specific binding probability $\pi$ has an uninformative Jeffreys prior given by a Beta distribution
%
\begin{gather*}
    \pi \in [0, 1] \\
    \pi \sim \mathbf{Beta}(1/2, 1/2)
\end{gather*}

\item The prior distribution for the spot presence indicator $m$ is conditional on $\theta$. When spot index $i$ corresponds to a target-specific spot, i.e., $\theta = i$, then $m_{\mathsf{spot}(i)} = 1$. When spot index $i$ does not correspond to a target-specific spot, i.e., $\theta \neq i$, then either there is a target-nonspecific spot corresponding to $i$ or a spot corresponding to $i$ does not exist. Consequently, for $\theta \neq i$ we assign $m_{\mathsf{spot}(i)}$ to either 0 or 1 with a probability dependent on the non-specific binding rate $\lambda$.
%
\begin{gather*}
    m_{\mathsf{spot}(i)}
    \begin{cases}
         = 1 & \text{if $\theta = i$} \\
        \sim \mathbf{Bernoulli} \left( \sum_{k=1}^K \dfrac{k \cdot \mathbf{TruncatedPoisson}(k; \lambda, K)}{K} \right) & \text{if $\theta = 0$} \\
        \sim \mathbf{Bernoulli} \left( \sum_{k=1}^{K-1} \dfrac{k \cdot \mathbf{TruncatedPoisson}(k; \lambda, K-1)}{K-1} \right) & \text{otherwise}
    \end{cases} \\
\end{gather*}

\noindent
The mean non-specific binding rate $\lambda$ is expected to be much less than two non-specifically bound spots per frame per AOI; therefore, we use an Exponential prior of the form
%
\begin{gather*}
    \lambda \in \mathbb{R}_{>0} \\
    \lambda \sim \mathbf{Exponential}(1)
\end{gather*}

\item The prior distribution for the integrated spot intensity $h$ is chosen to fall off at a value much greater than typical spot intensity values 
%
\begin{gather*}
    h \sim \mathbf{HalfNormal}(10000)
\end{gather*}

\item The optimal width $w$ of fluorescence spots in CoSMoS experiments is typically in the range of 1--2 pixels (ref). We use a Uniform prior confined to the range between 0.75 and 2.25 pixels
%
\begin{equation}
    w \sim \textbf{Uniform}(0.75, 2.25)
\end{equation}

\item Priors for spot position ($x$, $y$) depend on whether the spot represents target-specific or non-specific binding. Specifically bound molecules are co-localized with the target molecule with accuracy $\sigma^{xy}$ that is generally less than one pixel and depends on various factors including microscope point-spread function and magnification, accuracy of registration between binder and target channels, and accuracy of drift correction. We use an Affine-Beta prior with range constrained to one-half pixel beyond the image region, a mean position at the target molecule location $x_c$ and $y_c$, and a standard deviation parameterized as proximity $\sigma^{xy}$ (Figure S1b, orange). The specified range for ($x$, $y$) allows the center of off-target spots to fall up to one-half pixel beyond the AOI boundary. On the other hand, non-specific binding can occur anywhere within the image and therefore has a uniform distribution (Figure S1b, blue). The Uniform distribution is a special case of the Affine-Beta distribution.
%
\begin{gather*}
    x_{\mathsf{spot}(i)} \sim
    \begin{cases}
        \mathbf{AffineBeta}\left( x_c, \sigma^{xy}, -\dfrac{P+1}{2}, \dfrac{P+1}{2} \right) & \theta = i ~\textrm{(target-specific)} \\
        \mathbf{Uniform}\left(-\dfrac{P+1}{2}, \dfrac{P+1}{2} \right) & \theta \neq i ~\text{(target-nonspecific)}
    \end{cases} \\
    y_{\mathsf{spot}(i)} \sim
    \begin{cases}
        \mathbf{AffineBeta}\left( y_c, \sigma^{xy}, -\dfrac{P+1}{2}, \dfrac{P+1}{2} \right) & \theta = i ~\textrm{(target-specific)} \\
        \mathbf{Uniform}\left(-\dfrac{P+1}{2}, \dfrac{P+1}{2} \right) & \theta \neq i ~\text{(target-nonspecific)}
    \end{cases} \\
\end{gather*}

We give $\sigma^{xy}$ an Exponential prior with a characteristic width of one pixel.
%
\begin{gather*}
    \sigma^{xy} \in \mathbb{R}_{>0} \\
    \sigma^{xy} \sim \mathbf{Exponential}(1) \\
\end{gather*}

\item Gain $g$ depends on the settings of the amplifier and electron multiplier (if present) in the camera. It has a positive value and is typically in the range between 5--20. We use a Half-Normal prior with a broad distribution encompassing this range
%
\begin{gather*}
    g \in \mathbb{R}_{>0} \\
    g \sim \mathbf{HalfNormal}(50)
\end{gather*}

\item The prior distribution for the offset signal $\delta$ is empirically measured from the output of camera sensor regions that are masked from incoming photons. Collected data from these pixels are transformed into a density histogram with step size of $1$. Bin values and their weights are used to construct a Categorical prior
%
\begin{gather*}
    \delta \sim \mathbf{Categorical}(X, \dots, X)
\end{gather*}

\end{enumerate}

% \algrenewcommand\algorithmicrequire{\textbf{Model parameter:}}
\begin{algorithm}
\caption{Generative model for pixel intensities}
\label{alg:pseudocode}
\begin{algorithmic}[1]
\State $g \sim \mathbf{HalfNormal}(50)$
\Comment{camera gain}
\State $\sigma^{xy} \sim \mathbf{Exponential}(1)$
\Comment{std of on-target spot position (pixels)}
\State $\pi \sim \mathbf{Beta}(1/2, 1/2)$
\Comment{average specific binding probability}
\State $\lambda \sim \mathbf{Exponential}(1)$
\Comment{non-specific binding rate}
\ForAll{$\mathsf{AOI}[N]$}
\Comment{on-target data}
    \State $\mu^b \sim \mathbf{HalfNormal}(1000)$
    \Comment{mean background intensity}
    \State $\sigma^b \sim \mathbf{HalfNormal}(100)$
    \Comment{std of background intensity}
    \ForAll{$\mathsf{frame}[F]$}
    \Comment{time-independent}
        \State $b \sim \mathbf{Gamma}(\mu^b, \sigma^b)$
        \Comment{background intensity}
        \State $\theta \sim \mathbf{Categorical}\left(1 - \pi, \frac{\pi}{K}, \dots, \frac{\pi}{K}\right)$
        \Comment{target-specific spot index}
        
        \ForAll{$\mathsf{spot}[K]$}
            \State $ m_{\mathsf{spot}(i)}
                \begin{cases}
                    = 1 & \text{if $\theta = i$} \\
                    \sim \mathbf{Bernoulli} \left( \sum_{k=1}^K \dfrac{k \cdot \mathbf{TruncatedPoisson}(k; \lambda, K)}{K} \right) & \text{if $\theta = 0$} \\
                    \sim \mathbf{Bernoulli} \left( \sum_{k=1}^{K-1} \dfrac{k \cdot \mathbf{TruncatedPoisson}(k; \lambda, K-1)}{K-1} \right) & \text{otherwise}
                \end{cases} $
            \Comment{spot presence indicator}
            \If{$m=1$}
            \Comment{if spot is present}
                \State $h \sim \textbf{HalfNormal}(10000)$
                \Comment{spot intensity}
                \State $w \sim \textbf{Uniform}(0.75, 2.25)$
                \Comment{spot width}
                \State $ x_{\mathsf{spot}(i)} \sim
                    \begin{cases}
                    \mathbf{AffineBeta}\left( x_c, \sigma^{xy}, -\dfrac{P+1}{2}, \dfrac{P+1}{2} \right) & \theta = i \\
                    \mathbf{Uniform}\left(-\dfrac{P+1}{2}, \dfrac{P+1}{2} \right) & \theta \neq i \end{cases} $
                \Comment{$x$-axis center}
                \State $ y_{\mathsf{spot}(i)} \sim
                    \begin{cases}
                    \mathbf{AffineBeta}\left( y_c, \sigma^{xy}, -\dfrac{P+1}{2}, \dfrac{P+1}{2} \right) & \theta = i \\
                    \mathbf{Uniform}\left(-\dfrac{P+1}{2}, \dfrac{P+1}{2} \right) & \theta \neq i \end{cases}
                    $
                \Comment{$y$-axis center}
                \ForAll{$\mathsf{pixelX}[P] \times \mathsf{pixelY}[P]$}
                \State $\mu^{S} =
                            \dfrac{h}{2 \pi w^2} \exp{\left ( -\dfrac{(i-x)^2 + (j-y)^2}{2w^2} \right)}$
                \Comment{2D Gaussian spot}
                \EndFor
            \ElsIf{$m=0$}
                \State $\mu^S = 0$
            \EndIf
        \EndFor
            
        \ForAll{$\mathsf{pixelX}[P] \times \mathsf{pixelY}[P]$}
            \State $\delta \sim \mathbf{Empirical}( \delta_\mathrm{samples}, \delta_\mathrm{weights})$
            \Comment{offset signal}
            \State $\mu^I = b + \sum_{\mathsf{spot}[K]} m \odot \mu^S$
            \Comment{mean pixel intensity w/o offset}
            \State $I \sim \mathbf{Gamma} (\mu^I, \sqrt{\mu^I \cdot g})$
            \Comment{pixel intensity w/o offset}
            \State \Return $D = \delta + I$
            \Comment{observed pixel intensity}
        \EndFor
    \EndFor
\EndFor\algstore{myalg}
\end{algorithmic}
\end{algorithm}

% off-target control
\begin{algorithm}                     
\begin{algorithmic} [1]
\algrestore{myalg}
\ForAll{$\mathsf{AOI}[N_c]$}
\Comment{off-target control data}
    \State $\mu^b \sim \mathbf{HalfNormal}(1000)$
    \Comment{mean background intensity}
    \State $\sigma^b \sim \mathbf{HalfNormal}(100)$
    \Comment{std of background intensity}
    \ForAll{$\mathsf{frame}[F_c]$}
    \Comment{time-independent}
        \State $b \sim \mathbf{Gamma}(\mu^b, \sigma^b)$
        \Comment{background intensity}
        
        \ForAll{$\mathsf{spot}[K]$}
            \State $ m
                \sim \mathbf{Bernoulli} \left( \sum_{k=1}^K \dfrac{k \cdot \mathbf{TruncatedPoisson}(k; \lambda, K)}{K} \right) $
            \Comment{spot presence indicator}
            \If{$m=1$}
            \Comment{if spot is present}
                \State $h \sim \textbf{HalfNormal}(10000)$
                \Comment{spot intensity}
                \State $w \sim \textbf{Uniform}(0.75, 2.25)$
                \Comment{spot width}
                \State $ x \sim \mathbf{Uniform}\left(-\dfrac{P+1}{2}, \dfrac{P+1}{2} \right) $
                \Comment{$x$-axis center}
                \State $ y \sim \mathbf{Uniform}\left(-\dfrac{P+1}{2}, \dfrac{P+1}{2} \right) $
                \Comment{$y$-axis center}
                \ForAll{$\mathsf{pixelX}[P] \times \mathsf{pixelY}[P]$}
                \State $\mu^{S} =
                            \dfrac{h}{2 \pi w^2} \exp{\left ( -\dfrac{(i-x)^2 + (j-y)^2}{2w^2} \right)}$
                \Comment{2D Gaussian spot}
                \EndFor
            \ElsIf{$m=0$}
                \State $\mu^S = 0$
            \EndIf
        \EndFor
            
        \ForAll{$\mathsf{pixelX}[P] \times \mathsf{pixelY}[P]$}
            \State $\delta \sim \mathbf{Empirical}( \delta_\mathrm{samples}, \delta_\mathrm{weights})$
            \Comment{offset signal}
            \State $\mu^I = b + \sum_{\mathsf{spot}[K]} m \odot \mu^S$
            \Comment{mean pixel intensity w/o offset}
            \State $I \sim \mathbf{Gamma} (\mu^I, \sqrt{\mu^I \cdot g})$
            \Comment{pixel intensity w/o offset}
            \State \Return $D = \delta + I$
            \Comment{observed pixel intensity}
        \EndFor
    \EndFor
\EndFor
\end{algorithmic}
\end{algorithm}

\subsection*{Inference}

Combining the likelihood function and prior distributions we obtain the factorization of the joint probability distribution $p(D, \psi)$ of the model:
%
\begin{gather*}
    p(D, \psi) \\
    = \sum_\delta p(I | b, m, h, w, x, y, g) p(g) p(\delta) \\
    p(b | \mu^b, \sigma^b) p(\mu^b) p(\sigma^b) \\
    p(h | m) p(w | m) \\
    p(x | \sigma^{xy}, \theta, m) p(y | \sigma^{xy}, \theta, m) p(\sigma^{xy}) \\
    p(\theta | \pi) p(m | \theta, \pi, \lambda) p(\pi) p(\lambda) 
\end{gather*} 

\noindent
where $\psi = \{ b, \theta, m, h, w, x, y, g, \mu^b, \sigma^b, \pi, \lambda, \sigma^{xy} \}$.

For a Bayesian analysis, we want to obtain the posterior distribution for the parameters $\psi$ given the observed data using Bayes equation:
%
\begin{equation}
    p(\psi | D) =
    \dfrac{p(D, \psi)}{\int_\psi p(D, \psi) d\psi}
\end{equation}

Note that the integral in the denominator of this expression is necessary to calculate the posterior distribution, but it is usually analytically intractable. However, variational inference provides a robust method to approximate the posterior distribution $p(\psi | D)$ with a parameterized variational distribution $q(\psi)$ (ref).
%
\begin{gather*}
    p(\psi | D) \simeq q(\psi)
\end{gather*}


\subsubsection*{Variational distributions}

In order to apply variational inference, we must specify the parametric form of the variational distribution $q(\psi)$
that we use as an approximation to the true posterior distribution $p(\psi | D)$.

The variational distribution $q(\psi)$ has the following factorization with respect to model parameters:
%
\begin{gather*}
    q(\psi) = q(g) q(b) q(\mu^b) q(\sigma^b) \\
    q(h | m) q(w | m) \\
    q(x | m) q(y | m) q(\sigma^{xy}) \\
    q(\theta) q(m | \theta) q(\pi) q(\lambda)
\end{gather*}

Below we list variational distributions and their variational parameters.

\begin{enumerate}
    \item Background $b$
    \begin{gather*}
        b_\mathrm{mean} \in \mathbb{R}_{>0}^{\mathrm{AOI}[N] \times \mathrm{frame}[F]} \\
        b_\mathrm{beta} \in \mathbb{R}_{>0}^{\mathrm{AOI}[N] \times \mathrm{frame}[F]} \\
        b \sim \mathbf{Gamma}(b_\mathrm{mean}, b_\mathrm{beta})
    \end{gather*}
    
    Background mean $\mu^b$
    \begin{gather*}
        \mu^b_\mathrm{mean} \in \mathbb{R}_{>0}^{\mathrm{AOI}[N]} \\
        \mu^b \sim \mathbf{Delta}(\mu^b_\mathrm{mean})
    \end{gather*}
    
    Background standard deviation $\sigma^b$
    \begin{gather*}
        \sigma^b_\mathrm{mean} \in \mathbb{R}_{>0}^{\mathrm{AOI}[N]} \\
        \sigma^b \sim \mathbf{Delta}(\sigma^b_\mathrm{mean})
    \end{gather*}
    
    \item Intensity $h$
    \begin{gather*}
        h_\mathrm{mean} \in \mathbb{R}_{>0}^{\mathrm{spot}[K] \times \mathrm{AOI}[N] \times \mathrm{frame}[F]} \\
        h_\mathrm{beta} \in \mathbb{R}_{>0}^{\mathrm{spot}[K] \times \mathrm{AOI}[N] \times \mathrm{frame}[F]} \\
        h \sim \mathbf{Gamma}(h_\mathrm{mean}, h_\mathrm{beta})
    \end{gather*}
    
    \item Width $w$
    \begin{gather*}
        w_\mathrm{mean} \in [0.75, 2.25]^{\mathrm{spot}[K] \times \mathrm{AOI}[N] \times \mathrm{frame}[F]} \\
        w_\mathrm{size} \in \mathbb{R}_{>2}^{\mathrm{spot}[K] \times \mathrm{AOI}[N] \times \mathrm{frame}[F]} \\
        w \sim \mathbf{AffineBeta}(w_\mathrm{mean}, w_\mathrm{size}, 0.75, 2.25)
    \end{gather*}
    
    \item Positions $x$ and $y$
    \begin{gather*}
        x_\mathrm{mean} \in [-(P+1)/2, (P+1)/2]^{\mathrm{spot}[K] \times \mathrm{AOI}[N] \times \mathrm{frame}[F]} \\
        y_\mathrm{mean} \in [-(P+1)/2, (P+1)/2]^{\mathrm{spot}[K] \times \mathrm{AOI}[N] \times \mathrm{frame}[F]} \\
        xy_\mathrm{size} \in \mathbb{R}_{>2}^{\mathrm{spot}[K] \times \mathrm{AOI}[N] \times \mathrm{frame}[F]} \\
        x \sim \mathbf{AffineBeta} \left( x_\mathrm{mean}, xy_\mathrm{size}, -(P+1)/2, (P+1)/2 \right) \\
        y \sim \mathbf{AffineBeta} \left( y_\mathrm{mean}, xy_\mathrm{size}, -(P+1)/2, (P+1)/2 \right)
    \end{gather*}
    
    Proximity $\sigma^{xy}$
    \begin{gather*}
        \sigma^{xy}_\mathrm{mean} \in \mathbb{R}_{>0} \\
        \sigma^{xy}_\mathrm{beta} \in \mathbb{R}_{>0} \\
        \sigma^{xy} \sim \mathbf{Gamma}(\sigma^{xy}_\mathrm{mean}, \sigma^{xy}_\mathrm{beta})
    \end{gather*}
    
    \item Target-specific index parameter $\theta$
    \begin{gather*}
        \theta_\mathrm{prob} \in [0, 1]^{\mathrm{AOI}[N] \times \mathrm{frame}[F] \times \theta[K+1]} \\
        \theta \sim \mathbf{Categorical}(\theta_\mathrm{prob})
    \end{gather*}
    
    Average target-specific binding probability $\pi$
    \begin{gather*}
        \pi_\mathrm{mean} \in [0, 1] \\
        \pi_\mathrm{size} \in \mathbb{R}_{>2} \\
        \pi \sim \mathbf{Beta}(\pi_\mathrm{mean}, \pi_\mathrm{size})
    \end{gather*}
    
    \item Spot existence indicator $m$
    \begin{gather*}
        m_\mathrm{prob} \in [ 0, 1 ]^{\theta[K+1] \times \mathrm{spot}[K] \times \mathrm{AOI}[N] \times \mathrm{frame}[F]} \\
        m \sim \mathbf{Bernoulli}(m_\mathrm{prob})
    \end{gather*}
    
    Non-specific binding rate
    \begin{gather*}
        \lambda_\mathrm{mean} \in \mathbb{R}_{>0} \\
        \lambda_\mathrm{beta} \in \mathbb{R}_{>0} \\
        \lambda \sim \mathbf{Gamma}(\lambda_\mathrm{mean}, \lambda_\mathrm{beta})
    \end{gather*}
    
    \item Camera gain $g$
    \begin{gather*}
        g_\mathrm{mean} \in \mathbb{R}_{>0} \\
        g_\mathrm{beta} \in \mathbb{R}_{>0} \\
        g \sim \mathbf{Gamma}(g_\mathrm{mean}, g_\mathrm{beta})
    \end{gather*}
\end{enumerate}

\subsection*{Implementation}

The CoSMoS model and variational inference method outlined above are implemented as a probabilistic program, Tapqir, in the Python-based probabilistic programming language (PPL) Pyro \cite{Bingham2019-qy,Obermeyer2019-xt}. For a faster convergence we first fit the data to a model where $\theta$ is marginalize out and then fit to a full model to obtain $q(\theta)$.

\subsubsection*{Optimization}

In all of our analysis at each fitting iteration we use random subsample of AOIs as mini-batches where the size of the mini-batch is typically limited by the GPU memory available (Tapqir has a routine for finding maximum possible mini-batch size). We use PyTorch's Adam optimizer with the learning rate of $5\times 10^{-3}$ and keep other parameters at their default values. As a stopping criteria we use a heuristic rule

\begin{gather*}
    \dfrac{\mathrm{std}_\mathsf{iteration}(\mathrm{param}_{\mathsf{iteration}(-10000:0:100)})}{\mathrm{std}_\mathsf{iteration}(\mathrm{param}_{\mathsf{iteration}(-5000:0:100)})} < 1.05
\end{gather*}

\noindent where $\mathrm{param} = \{ \mathrm{ELBO}, g_\mathrm{mean}, \sigma^{xy}_\mathrm{mean}, \pi_\mathrm{mean}, \lambda_\mathrm{mean} \}$

\subsection*{Data simulation}

For each simulation, the size of the dataset ($D=14$, $N$, $F$, $N_c$, $F_c$) was fixed and a subset of parameters ($K=2$, $\pi$, $\lambda$, $g$, $\sigma^{xy}$, $b$, $h$, $w$, $\delta$) were set to desired values.  The remaining parameters ($\theta$, $m$, $x$, $y$) and resulting noisy images ($D$) were produced using Tapqir's generative model. Chosen parameter values and dataset sizes are provided in Supplementary Information.

For kinetic simulations, $\theta$ was modeled using a discrete Markov process. For $K=2$, $\theta$ has three states, with $\theta = 0, 1, 2$ corresponding to no specifically bound spot being present, spot 1 being specifically bound, and spot 2 being specifically bound, respectively. When a specifically bound molecule is present (i.e., $\theta = 1, 2$), the probability is equally split between those indices. We assume that the Markov process is at equilibrium and initialized the chain with the equilibrium probabilities.

% The probability that $\theta = 0$ is equal to the probability of no specifically bound spot being present (i.e., $1-\pi$). Since spot indices are arbitrarily assigned,   (i.e., $\frac{\pi}{K}$).

\begin{gather*}
    \mathsf{init} \in [0, 1]^{\theta[K+1]} \quad \sum_{\theta} \mathsf{init} = 1 \\
    \mathsf{init} = \begin{pmatrix} \frac{k_\mathrm{off}}{k_\mathrm{on} + k_\mathrm{off}} & \frac{k_\mathrm{on}}{2\left( k_\mathrm{on} + k_\mathrm{off} \right)} & \frac{k_\mathrm{on}}{2\left( k_\mathrm{on} + k_\mathrm{off} \right)} \end{pmatrix} \\
    \mathsf{trans} \in [0, 1]^{\theta^\prime[K+1] \times \theta[K+1]} \quad \sum_{\theta} \mathsf{trans} = 1 \\
    \mathsf{trans} = \begin{pmatrix} 1 - k_\mathrm{on} & k_\mathrm{on}/2 & k_\mathrm{on}/2 \\ k_\mathrm{off} & (1 - k_\mathrm{off})/2 & (1 - k_\mathrm{off})/2 \\ k_\mathrm{off} & (1 - k_\mathrm{off})/2 & (1 - k_\mathrm{off})/2 \end{pmatrix} \\
    \theta_{\mathsf{frame}(1)} \sim \mathrm{Categorical(\mathsf{init})} \\
    \theta_{\mathsf{frame}(t)} \sim \mathrm{Categorical(\mathsf{trans}_{\theta^\prime( \theta_{\mathsf{frame}(t-1)})})}
\end{gather*}

\noindent
where $\mathsf{init}$ is a vector of equilibrium probabilities for $\theta$, $\mathsf{trans}$ is the transition probability matrix, and $k_{\mathrm{on}}$ and $k_{\mathrm{off}}$ are the apparent first-order binding and dissociation rate constants in units of $\mathrm{s}^{-1}$, respectively, assuming 1 s/frame.

For posterior predictive checking, simulated images were produced using Tapqir's generative model where model parameters were sampled from the variational distribution $q(\psi)$. % reference Figure S3 (sampling from posterior of simulated data)

\subsection*{Classification accuracy statistics}

As a metric of classification accuracy we use three commonly used statistics -- recall, precision, and Matthews Correlation Coefficient \cite{Matthews1975-rw}
\begin{gather*}
    \mathrm{Recall} = \dfrac{\mathrm{TP}}{\mathrm{TP} + \mathrm{FN}}
\end{gather*}

\begin{gather*}
    \mathrm{Precision} = \dfrac{\mathrm{TP}}{\mathrm{TP} + \mathrm{FP}}
\end{gather*}

\begin{gather*}
    \mathrm{MCC} =
        \dfrac{\mathrm{TP} \cdot \mathrm{TN} - \mathrm{FP} \cdot \mathrm{FN}}
        {\sqrt{(\mathrm{TP} + \mathrm{FP}) (\mathrm{TP} + \mathrm{FN}) (\mathrm{TN} + \mathrm{FP}) (\mathrm{TN} + \mathrm{FN})}}
\end{gather*}

\noindent
where TP is true positives, TN is true negatives, FP is false positives, and FN is false negatives (ref).

\subsubsection*{SNR}

We calculate signal-to-noise ratio (SNR) as the mean of ratio of signal to noise for each spot

\begin{gather*}
    \mathrm{SNR} = \mathrm{mean} \left( \dfrac{\mathsf{signal}}{\mathsf{noise}} \right)
\end{gather*}

Spot signal is defined as weighted average of total signal minus mean background and mean offset signals

\begin{gather*}
    \mathsf{weight} = \dfrac{1}{2 \pi \cdot w^2} \exp{\left( -\dfrac{(i-x)^2 + (j-y)^2}{2 \cdot w^2} \right)} \\
    \mathsf{signal}_{\mathsf{spot}(k)} =  \sum_{\substack{\mathsf{pixelX} \\ \mathsf{pixelY}}} D \cdot \mathsf{weight}_{\mathsf{spot}(k)} - b_{\mathrm{mean}} - \delta_\mathrm{mean} \quad \textrm{for } p(\theta = k) > 0.5
\end{gather*}

\noindent and noise as

\begin{gather*}
    \mathsf{noise}^2 = \sigma^2_{\mathsf{offset}} + b_\mathrm{mean} \cdot g_\mathrm{mean}
\end{gather*}

For simulated data theoretical signal can be directly calculated as

\begin{gather*}
    \mathsf{signal}_{\mathsf{spot}(k)} =  \sum_{\substack{\mathsf{pixelX} \\ \mathsf{pixelY}}} h \cdot \mathsf{weight}_{\mathsf{spot}(k)}^2 \quad \textrm{for } k = \theta
\end{gather*}

\subsection*{Kinetic analysis}

Our procedure to estimate kinetic parameters is as follows (see Algorithm 2). For each iteration we sample binary data records $z$ from inferred $p(\mathsf{specific})$. Then we compute dwell time $\Delta t$ (binding-present -- $\Delta t_\mathrm{on}$, binding-absent -- $\Delta t_\mathrm{off}$, time-to-first binding -- $\Delta t_\mathrm{ttfb}$) and maximum-likelihood estimate of parameter $\hat{k}$ based on $\Delta t$. After completing 1000 iterations we compute mean and confidence interval from the distribution of $\hat{k}$.

For single-exponential kinetics used in all of our analyses maximum-likelihood estimate $\hat{k}$ is given by

\begin{gather*}
    \Delta t \in \mathbb{R}_{>0}^{\mathsf{event}[M]} \\
    p(\Delta t | k) = \prod_\mathsf{event} k \exp (- k \Delta t) \\
    \hat{k} = \dfrac{1}{\mathrm{mean}_{\mathsf{event}} \Delta t}
\end{gather*}

\begin{algorithm}
\caption{Generative model for pixel intensities}
\begin{algorithmic}[1]
\For{$i = 1, 2, \dots, 1000 $}
    \State Sample binary $z$ from approximate posterior $p(\mathsf{specific})$
    \State Calculate $\Delta t$ from $z$
    \State Calculate $\hat{k}_i$ from $\Delta t$
\EndFor{}
\State Calculate mean and CI from the distribution of $\hat{k}$
\end{algorithmic}
\end{algorithm}
\section*{Results}

\subsection*{Observed CoSMoS data} % Figure 1B,C,D

To maximize the extraction of useful information from data we use full 2-D images for analysis. For co-localization analysis it is sufficient to analyze the image area local to the target molecule. The area of interest (AOI) analyzed needs to be large enough so that fluorescence spots can be clearly distinguished from background, allowing  reliable  determination of the  local  background  intensity. Typical half-width of fluorescence spots in experiments used in this work is 1-2 pixels and therefore we chose the size of AOI to be $14\times14$ pixels (Fig. 1c,d). In addition to AOIs centered at target molecules, it is useful to also collect AOI images from randomly selected sites at which no target molecule is present as a negative control data (Fig. 1d). Such off-target control data is analyzed jointly with on-target data and serves to facilitate the estimation of the background level of target-nonspecific binding. Construction of CoSMoS data for Tapqir includes localization of target molecules in the target channel, registration of target and binder channels, and time-dependent drift correction and can be performed with existing tools \cite{Friedman2015-nx, Smith2019-yb} (Fig. 1b).

 % One experiment typically consists of a set of images where we have $N$ target sites ($n \in \{1,\dots,N\}$) each consisting of a series of $F$ different images in a recording ($f \in \{1,\dots,F\}$) (a “recording”).

\subsection*{Bayesian image classification analysis}

To solve the problems with existing CoSMoS data analysis identified above, we developed a new method that is accurate, objective, and built on a rigorous Bayesian statistical approach to the CoSMoS image analysis problem. To perform a Bayesian analysis of the data one must 1) define the model for the observed data in terms of model parameters, 2) specify priors for random model parameters, and 3) specify the likelihood function for the observed data given the model. Our CoSMoS model classifies the observed images using a probabilistic mixture formalism that includes spots from both specifically and non-specifically bound molecules. Specification of priors for random model parameters allows us to embed our knowledge about the experiment (e.g., the likely positions of specifically and non-specifically bound molecules). Our likelihood function takes the idealized image model and incorporates physically realistic photon shot-noise. Inference of model parameters and classification of images is summarized in posterior distributions which are calculated using Bayes' rule. The analysis is “time-independent”, meaning that we ignore the time dimension of the recording -- the order of the images does not affect the model, as each image is considered statistically independent of the others. 

\subsection*{Probabilistic model structure and parameters} % Figure 2A,B,C

A single AOI image from a CoSMoS dataset is a matrix of noisy pixel intensity values centered at the location of a target molecule. In each image, multiple binder molecules can be present. Fig. 2a shows an example image where two spots are present, one spot is located near the target molecule at the center of the AOI and another is off-target. We model an AOI image as a constant average background intensity $b$ summed with fluorescence spots modeled as 2-D Gaussians $\mu^S$, which accurately approximate the microscope point spread function \cite{Zhang2007-rb} (Fig. 2b). The spots are numbered and each Gaussian is parameterized by integrated intensity $h$, width $w$, and position $x$ and $y$ relative to the AOI center. For the data we typically encounter, there are no more than two spots present in a single AOI. We therefore use $K$ = 2 as the default maximum  number of spots per AOI in all analyses presented here. The presence or absence of each spot is denoted by a binary indicator parameter $m \in \{0, 1\}$. The resulting mixture model has four possible combinations for $m_{\mathsf{spot}(1)}$ and $m_{\mathsf{spot}(2)}$: (1) a no-spot image that contains only background (Fig. 2b, top left), (2) a single-spot image that contains the first binder molecule spot superimposed on background (Fig. 2b, bottom left), (3) a single-spot image that contains the second binder molecule spot superimposed on background (Fig. 2b, top right), and (4) a two-spot image that contains both binder molecule spots superimposed on background (Fig. 2b, bottom right).

Among the spots that are present in an AOI, by assumption at most only one can be target-specific. To classify spots as specific or non-specific we introduce the index parameter $\theta \in \{0,1,\dots,K\}$. Index $\theta$ uniquely determines specifically bound molecule when it is present (Fig. 2c, middle and right) and equals zero when it is absent (Fig. 2c, left). Since negative control AOIs which are not centered on target molecules contain only non-specific binding, we set $\theta = 0$ for all control data images. 

The resulting probabilistic model can be interpreted as a generative process that produces the observed image data. A graphical representation of the probabilistic relationships in the model is shown in Fig. 2d (see also Extended Data Fig. 1a).

\subsection*{Parameter prior distributions}

Specifying prior distributions for model parameters is essential for Bayesian analysis and allows us to incorporate pre-existing knowledge about experimental design. For most model parameters, there is no strong prior information so we use uninformative prior distributions (see Methods). However, we have strong expectations for the positions of specific and non-specific binder molecules that can be expressed as prior distributions and used effectively to discriminate between the two. Non-specific binding can occur anywhere on the surface with equal probability and thus has a uniform prior distribution across the image. Target-specific binding, on the other hand, is co-localized with the target molecule and thus has a prior distribution peaked at the AOI center (Extended Data Fig. 1b). The width of the prior distribution for specific binding location, proximity parameter $\sigma^{xy}$, depends on multiple experiment characteristics such as the spot localization accuracies and the mapping accuracy between target and binder channels. Prior distributions for parameters $\theta$ and $m$ are defined in terms of average specific binding probability $\pi$ and non-specific binding rate $\lambda$. Since by definition specific binding cannot occur in control data, we set $\pi = 0$ for that data. To allow convenient use of the algorithm, we infer values of $\sigma^{xy}$, $\pi$, and $\lambda$ appropriate to a given dataset using a hierarchical Bayesian analysis (see Methods and Extended Data Fig. 1a). 

\subsection*{Likelihood function and image noise}

Each measured pixel intensity in a single-molecule fluorescence image has a noise contribution from photon counting (shot noise) and can also have additional noise from electronic amplification (ref?). The result is  a characteristic linear relationship between the noise variance and  mean intensity with a slope defined as the image gain $g$. The model's likelihood function incorporates a realistic noise model that accounts for both pixel noise and offset (see Methods). 

\subsection*{Bayesian inference and implementation}

Posterior distributions of model parameters conditioned on the observed data are obtained by using Bayes' theorem. We use a stochastic variational inference (SVI) approach to approximate the posterior distributions and to infer model parameters. Our Bayesian SVI algorithm and model are implemented as a program, Tapqir, in the Python-based probabilistic programming language Pyro \cite{Bingham2019-qy}. Complete details of the model and implementation are given in the Methods section.

\subsection*{Tapqir analysis} % Figure 3A, B

To test the approach, we used Tapqir to analyze simulated CoSMoS image data with high SNR (3.76) as well as data from the experiment shown in Fig. 1, which has lower SNR (XXX). Data were simulated using the same generative model that was used for analysis (Fig 2d) with chosen subset of parameters held at fixed values (Table S3). For both experimental and simulated datasets, on-target and control data were analyzed jointly. For both simulated data (Figure~\ref{fig:tapqir_analysis}A) and experimental data (Figure~\ref{fig:tapqir_analysis}B), Tapqir correctly detects and locates fluorescent spots in the images (Figure~\ref{fig:tapqir_analysis}, compare AOI images and spot detection). When a spot is present, corresponding to a high $p(m)$ (e.g., spot 1 (blue) in frame 630 in Figure 3b) it also has a high intensity parameter ($h$) and a comparatively small uncertainty in position coordinates $x$ and $y$. In contrast, when the spot is absent (e.g., both spots 1 and 2 in frame 164 in Figure~\ref{fig:tapqir_analysis}a) it has a near zero intensity and high uncertainty in position because no spot can be localized. The program also determines the background intensity $b$ for each image without requiring a separate measurement. 

Tapqir calculates the probability $p(\theta=1)$ that a spot represents a molecule specifically bound to the target, based on whether the spot is colocalized with the target molecule located at the center of the AOI.  For example, spot 1, frame 170 in Figure~\ref{fig:tapqir_analysis}a and spot 1, frame 633 in Figure~\ref{fig:tapqir_analysis}b both are classified as specific ($p(\theta) \approx 1$) while both spots in frame 160 of Figure 3a are classified as non-specific ($p(\theta) \approx 0$). Apart from these two extreme scenarios, intermediate cases can also be observed in the noisy experimental data (e.g., spot 1, frame 635 in Figure 3B). Spot classification is summarized as the $p(\mathsf{specific})$ curve in Figure~\ref{fig:tapqir_analysis}, which is calculated from $p(\theta)$ and corresponds to the probability of there being a target-specific spot in the AOI. 

%\subsection*{Validation with posterior sampling}
% Figure 4

To evaluate how well the model fit the data, we simulated AOI images from the posterior distributions of parameters (a method known as posterior predictive checking) (ref). The posterior predictive simulations accurately reproduce the experimental AOI appearances, recapitulating the noise characteristics and the numbers, intensities, shapes, and locations of spots (Figure~\ref{fig:posterior_samples}).  The distributions of pixel intensities across the AOI are also closely reproduced (Figure 4, histograms) confirming that the noise model is accurate. Taken together, these results show that the model is rich enough to accurately capture the full range of image characteristics from CoSMoS datasets taken over different experimental conditions.

\subsection*{Tapqir performance}
% Figure S2 

Next, we evaluated Tapqir's ability to reliably infer the values of global model parameters. To  accomplish this, we generated simulated datasets  using  a wide range of randomized parameter values and then fit the simulated data to the model (Table S2). Fit results show that global model parameters (i.e., average specific spot probability $\pi$, nonspecific binding rate $\lambda$, proximity $\sigma^{xy}$, gain $g$) are close to the simulated values  (Figure S2 and Table S2). This suggests that CoSMoS data contains enough information to reliably infer global model parameters and that the model is not obviously overparameterized.

Having tested the basic function of the algorithm, we next turned to the key question of how accurately Tapqir detects target-specific spots in increasingly difficult datasets with either decreasing SNR or with increasing frequencies of target-nonspecific binding.  

First, we examined how the accuracy of target-specific spot detection varies with  image SNR using simulated datasets (Table S3). By eye, spots can be readily discerned at SNR $>1$ but cannot be clearly seen at SNR $<1$ (Figure 5a). Tapqir gives similar or better performance:  if an image is known to contain a target-specific spot, Tapqir assigns it target-specific spot probability $p(\mathsf{specific})$ which is on average close to  one when SNR $>1$ (Figure 5b).  This value sharply decreases below SNR 1, consistent with the subjective impression that no spot is seen under those conditions.  The images that contain a target-specific spot (Figure 5c, green) are almost always assigned a high $p(\mathsf{specific})$ for high SNR data and almost always assigned low $p(\mathsf{specific})$ for low SNR data.  At SNR $\simeq 1$, these images are assigned a broad distribution of $p(\mathsf{specific})$, accurately reflecting the uncertainty in classifying such data.  Just as importantly, images with no target-specific spot are almost always assigned $p(\mathsf{specific}) < 0.5$ correctly reflecting the absence of the spot.

Ideally, we want to correctly identify target-specific binding when it occurs but also avoid identifying target-specific binding when it does not occur. To quantify Tapqir's classification accuracy in a way that accounts for true and false classifications of frames, we next examined binary image classification statistics. We thresholded $p(\mathsf{specific})$ at $0.5$ and calculated three statistics: \textit{recall}, \textit{precision}, and the Matthews Correlation Coefficient (MCC) \cite{Matthews1975-rw} (Figure 5d; see Methods). \textit{Recall} is defined as the fraction of simulated target-specific spots that are correctly predicted. Recall is a binary version of $p(\mathsf{specific})$ and as expected has a similar shape. \textit{Precision} is the fraction of all target-specific spot detections that are correct. Precision is near one at all SNR values tested; this shows that the algorithm rarely misclassifies an image as containing a target-specific spot when none is present. At the lowest SNR value (0.38) only a tiny fraction of images are assigned $p(\mathsf{specific})$ $>$ 0.5 (Figure 5c, leftmost panel), but within this small subset all contain a target-specific spot and thus are correctly identified. The MCC is a single value that compares whether or not a target-specific spot is predicted by Tapqir vs. whether or not a target-specific spot is actually present in the simulated data.  The MCC is equivalent to the Pearson correlation coefficient between the predicted and true classifications, giving 1 for a perfect match, 0 for a random match, and -1 for complete disagreement. The MCC results suggests that the overall performance of Tapqir is excellent at SNR $\ge 1$: the program rarely misses target-specific spots that are in reality present and rarely falsely reports a target-specific spot when none is present.  

The forgoing analyses (Figure 5b-d) examined Tapqir performance on data in which the rate of target-nonspecific binding was moderate ($\lambda  = 0.15$ non-specific spots per image on average).  We next examined the effects of increasing the non-specific rate.  Specifically, we used  simulated data with SNR = 3.76 to test the classification accuracy of Tapqir at different non-specific binding rates up to $\lambda  = 1.0$ non-specifically bound molecule per image on average, a rate considerably higher than typical in experimental data (compare experimental datasets in Table S1).  Even at the highest $\lambda$,  Tapqir accurately identified target-specific spots (Figure 5e,f) and returned good binary classification statistics (Figure~\ref{fig:tapqir_performance}g).  At the highest non-specific spot rates, a few images with target-specific spots are classified as not having a specific spot ($p(\mathsf{specific})$ near zero) or as being ambiguous ($p(\mathsf{specific})$ near 0.5) (Figure 5f, green bars). Similarly, a few images with target-nonspecific spots are classified as having specific spot ($p(\mathsf{specific})$ near or above 0.5) which only occurs at the highest $\lambda$ (Figure 5f, gray bars).

We also tested the opposite extreme by simulating data sets with no target-specific binding at both low and high non-specific binding rates. Analysis of such data  (Figure 5h) shows that no target-specific binding ($p(\mathsf{specific}) > 0.5$) was detected even under the highest non-specific binding rate, demonstrating that Tapqir is robust to false-positive target-specific spot detection even under these extreme conditions. 

\subsection*{Kinetic and thermodynamic analysis}
% Figure 6

The most widespread application of CoSMoS experiments is to measure rate and equilibrium constants for the interaction of the target and binder molecules being studied.  We next tested whether these constants can be accurately determined using Tapqir-calculated posterior parameters. 

We first simulated CoSMoS datasets that reproduced the behavior of a one-step association/dissociation reaction mechanism (Figure 6a). Simulated data (e.g., Fig. 6b, purple) were analyzed with Tapqir.   As discussed previously, an advantage of Tapqir is that, unlike current CoSMoS data analysis methods, it accounts for the reliability of the data by giving the posterior probability $p(\mathsf{specific})$ of a target-specific spot in each image (e.g., Fig. 6b, green) rather than merely producing a binary spot/no-spot output.  We use the posterior probability time series in kinetic analysis by employing a Monte Carlo sampling method that naturally provides both typical values and uncertainties of the kinetic rate constants.  In particular, we sample a family of binary data records (Fig. 6b, black) from $p(\mathsf{specific})$, each of which has well-defined target-specific binder-present and binder-absent intervals $\Delta t_\mathrm{on}$ and $\Delta t_\mathrm{off}$. These intervals are then fit to exponential functions (see Methods) to determine both the values and the uncertainties (i.e., standard errors) of $k_\mathrm{on}$ and $k_\mathrm{off}$ (Figure~\ref{fig:kinetic_analysis}c,d). Comparison of the simulated and estimated values shows that both rate constants are accurate within 30\% at nonspecific binding rates typical of experimental data ($\lambda \leq 0.5$). At higher nonspecific binding rates, rare interruptions caused by false-positive and false-negative spot detections shorten $\Delta t_\mathrm{on}$ and $\Delta t_\mathrm{off}$ distributions, leading to moderate systematic overestimation of the association and dissociation rate constants.

We used the same simulations to estimate the equilibrium dissociation constant $K_\mathrm{eq} = k_\mathrm{on}/k_\mathrm{off}$.  Calculation of the equilibrium constant and its uncertainty does not require a time-dependent model and can be obtained directly from the posterior distribution of the specific-binding probability $\pi$. The estimated equilibrium constants are highly accurate even at excessively  high values of $\lambda$ (Figure 6e).  The high accuracy results from the fact that equilibrium constant measurements are much less affected than kinetic measurements by occasional errors in spot detection. 

% Figure 7
Experimental CoSMoS data sets are diverse.  In addition to having different SNR and non-specific binding frequency values, they also may have non-idealities in spot shape caused by optical aberrations and in noise caused by molecular diffusion in and out of the TIRF evanescent field.  In order to see if Tapqir analysis is robust to these and other properties of real experimental data, we  analyzed several CoSMoS data sets taken from different experimental projects (Table S1, Figure 7, Figures S3 - S4). We first visualized the results as rastergrams (Figures 7a, S3a, and S4a), in which each horizontal line represents the time record from a single AOI.  Unlike the binary spot/no-spot rastergrams in previous studies (refs?) we plotted the Tapqir-calculated spot probability $p(\mathsf{specific})$ using a color scale.  This representation allows a more nuanced understanding of the data.  For example, Fig. 7a reveals that while the long-duration spot detection events typically are assigned a high probability (yellow), some of the shortest duration events have an intermediate $p(\mathsf{specific})$ (green) indicating that the assignment of these as target-specific is uncertain.  

To demonstrate the utility of Tapqir for kinetic analysis of experimental datasets, we analyzed the durations of the binder-absent intervals that preceded the first binding event.  Such time-to-first binding analysis improves accuracy by minimizing the effect of target molecules occupied by photobleached binders \cite{Friedman2006-kb}.  Specifically, the analysis used the Monte Carlo posterior sampling method (as in Figure 6b, black records) to determine the initial  $\Delta t_\mathrm{off}$ in each record and then fit the distribution of these values to calculate the target-specific association rate constant $k_\mathrm{a}$, as well as the non-specific association rate $k_\mathrm{ns}$ and the active fraction of target molecules $A_\mathrm{f}$ (Figs. 7b, S3b, and S4b). Comparison to the results of a previously published method that analyzes images using an empirical binary spot-picking method \cite{Friedman2006-kb} (Figures 7c, S3c, and S4c) shows that values of parameters obtained using these two methods are in reasonable agreement with each other. We emphasize that results obtained by Tapqir and the spot-picker method are not expected to agree exactly since they are based on different models.  For example, Tapqir accounts for the possibility that target-non specific and target-specific spots can be present simultaneously in the same image, and is based on a specific model of image noise.  Tapqir provides information about the reliability of the spot classification in each image and does so without manual fine-tuning of analysis parameters for different data sets, allowing reliable determination of molecular kinetic and thermodynamic properties.


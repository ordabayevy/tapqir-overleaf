\section*{Discussion}

%We here describe an algorithm and computer software, Tapqir, for rigorous statistical classification and analysis of image data from single-molecule fluorescence colocalization experiments. The approach is distinct from existing methods in multiple ways that convey important advantages for the accuracy and robustness of CoSMoS data analysis: 1) it uses a global model that captures all the important physical aspects of CoSMoS data generation in a single unified framework, 2) it employs Bayesian inference, incorporating appropriate levels of prior knowledge for all model parameters, 3) it produces spot probability estimates that can be used to inform downstream kinetic and thermodynamic analyses, and 4) it is implemented in a probabilistic programming language that facilitates enhancement, modification, and analysis of the model. Tapqir performs as well as or better than  existing high-quality CoSMoS data analysis methods. It achieves this performance automatically, without requiring the subjective tuning of analysis parameters that makes effective use of other methods challenging to non-expert users.
%also: off-target handling is different

% probabilistic modeling
A broad range of physical processes contribute to the formation of CoSMoS images. These include camera and photon noise, target-specific and non-specific binding, and time- and position-dependent variability in fluorophore imaging and image background. Unlike prior CoSMoS analysis methods, Tapqir considers these aspects of imaging in a single, holistic model. The model explicitly includes the uncertainties due to photon noise, camera gain, and spatial variability in intensity offset. The model also includes the possibility of multiple binder molecule fluorescence spots being present in the vicinity of the target, including both target-specific binding and target-nonspecific interactions of binder molecules with the coverslip surface. This explicit modeling of target-nonspecific spots makes it possible to include off-target control data as a part of the experimental data set.  Similarly, all AOIs and frames in the data set are simultaneously fit to the global model in a way that allows for realistic frame-to-frame and AOI-to-AOI variability in image formation caused by variations in laser intensity, fluctuations in background, and other non-idealities. The global analysis based on a single, unified model enables the final results (e.g., kinetic and thermodynamic parameters) to be estimated in a way that is cognizant of all known sources of uncertainty in the data. 

Previous approaches to CoSMoS data analysis, including ours, did not employ a holistic modeling approach and instead relied on a separate binary classification step.  These prior methods require subjective setting of classification thresholds.  Because they are not fully objective, such methods cannot reliably account for uncertainties in spot classification, which compromises error estimates in the analysis pipeline downstream of spot classification. One previous approach \cite{Smith2019-yb,Smith2015-gf}, which like Tapqir analyzes 2-D images instead of integrated intensities, used a Bayesian kinetic analysis but did not classify images using Bayesian methodology. It instead used a frequentist hypothesis test (a generalized likelihood ratio test) for spot detection, which lacks the key advantages of the model-based Bayesian analysis used here.  The most significant of these advantages is that Tapqir objectively predicts target-specific spot presence probabilities $p(\mathsf{specific})$ for each image, rather than the binary ``spot/no spot'' classification used in prior approaches. In essence, previous approaches assume that spot classifications are correct, and thus the uncertainties in the derived molecular properties (e.g., equilibrium constants) are systematically underestimated because the (often large) errors in spot classification are not accounted for. By performing a probabilistic spot classification, Tapqir enables statistically reliable estimates of molecular properties such as thermodynamic and kinetic parameters.  This more inclusive error estimation likely accounts for the generally larger kinetic parameter error bars seen for Tapqir vs. the spot-picker analysis (Fig. 7 and Extended Data Figs. 4-6).

% Bayesian inference
Our model includes parameters of mechanistic interest, such as the average probability of target-specific binding, as well as ``nuisance'' parameters that are not of primary interest but nevertheless essential for image modeling. In previous image-based methods for CoSMoS analysis (e.g., \cite{Friedman2015-nx,Smith2019-yb}), nuisance parameters were either measured in separate experiments (e.g., gain was determined from calibration data), set heuristically (e.g., a subjective choice of user-set thresholds for spot intensity and proximity in colocalization detection), or determined at a post-processing analysis step (e.g., rate of non-specific binding). In contrast, Tapqir directly learns these parameters from the full set of experimental data, thus eliminating the need for additional experiments, subjective adjustment of tuning parameters, and post-processing steps.

A key feature of Bayesian analysis is that the extent of prior knowledge of all model parameters is explicitly incorporated. Where appropriate, Tapqir uses relatively uninformative priors that only weakly specify information about the value of the corresponding parameters.  In these cases, the program mostly infers parameter values from the data.  In contrast, some priors are more informative.  For example, binder molecule spots near the target molecule are more likely to be target-specific rather than target-nonspecific, so we use this known feature of the experiment by encoding the likely position of target-specific binding as a Bayesian prior. This tactic effectively enables probabilistic classification of spots as either target-specific or target-nonspecific, which would be difficult using other inference methodologies.

% probabilistic programming
Tapqir is implemented in Pyro, a Python-based probabilistic programming language (PPL) \cite{Bingham2019-qy}. Probabilistic programming is a relatively new paradigm in which probabilistic models are expressed in a high-level language that allows easy formulation, modification, and automated inference \cite{Van_de_Meent2018-mi}. In this work we focused on developing an image model for colocalization detection in a relatively simple binder-target single-molecule experiment. However, Tapqir can be extended readily to more complex models. For example, we expect Tapqir to naturally extend to multi-state and multi-color analysis. Furthermore, with the development of more efficient sequential hidden Markov modeling algorithms \cite{Sarkka2019-jw,Obermeyer2019-pp} Tapqir can potentially be extended to directly incorporate kinetic processes, allowing direct inference of kinetic mechanisms.
%Design principles and use of GPU-supported backend numeric libraries in modern PPLs such as Pyro enable flexible and expressive probabilistic modeling scalable to large data sets. Scalability is a significant consideration, since the sizes of CoSMoS data sets have increased due to faster, larger cameras \cite{Quan2011-cg}, the availability of fluorescent dyes with dramatically improved photostability \cite{Grimm2015-ea} (ref. JF group, Blanchard) (which increase the amount of data from a single experiment), and microscopes that efficiently collect single-molecule data at more wavelengths simultaneously \cite{Friedman2006-kb}.

% Keep for cover letter: It is a niche tool used previously by statisticians (refs.) and to some extent in bioinformatics (refs) but has not been widely used to analyze laboratory data in the life sciences. 

Tapqir and Pyro are open-source software; Tapqir and its documentation are available at \url{https://github.com/gelles-brandeis/tapqir}.

% possible reviewers:  Ruben Gonzales, Jan van de Meent, Chris Wiggins, Taekjip Ha, ?David Grunwald, ?Carlas Smith, ?Aaron Hoskin, Shixin Liu (Rockefeller)
% Possible info for cover letter: Like other recent methods for CoSMoS analysis\cite{Friedman2015-nx,Smith2019-yb}, Tapqir maximizes extraction of useful information from data (particularly at the low S/N inherent to many CoSMoS experiments) by directly analyzing 2-D images as opposed to integrated fluorescence. Like Tapqir, these image analysis based methods can be more accurate than previous approaches in part because they make use of information contained in the 2-D images that are disregarded when only the integrated intensity is used\cite{Friedman2015-nx}. While these methods greatly improve CoSMoS data analysis relative to methods based on integrated fluorescence intensity records, they nevertheless suffer from significant deficiencies. First, current methods require subjective choice of user-set thresholds for spot amplitude, diameter and proximity. These settings significantly affect error rates and no objective method to select them currently exists, making the approach non-robust and overly complicated. Second, it is unsatisfying from a theoretical perspective that existing analysis methods are performed in multiple steps with a loss of information about the uncertainties at each step of the analysis. Spot detection is performed on the extracted features of the spot and not the raw 2-D images themselves. Furthermore, spot detection step produces only a binary output (spot present or absent); they do not output the probability of spot presence in marginal images, a critical feature in the analysis of low S/N data. Kinetic analysis, in turn, is based on the binary output of the spot detection step, further reducing the amount of information transmitted from the raw images. Third, it is nontrivial to incorporate prior knowledge (e.g., colocalization accuracy, size of the spot, frequency of non-specific binding) into these analysis methods.

\begin{comment}

NEW TODO LIST
- Update Abstract
- swap x and y in fig3
- coordinate system figure
- remake fig4 in python
- remake fig6 in python
- remake fig7, ed fig 4,5,6 in python and edit captions
- Extended Data Table 1: formatting, confidence ints., computation time
- Eq 31
- Edit paragraph two in Methods
- Extended Data Figure 2 caption
--on publication --
Data archive

\end{comment}

